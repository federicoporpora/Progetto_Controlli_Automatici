\documentclass[a4paper, 11pt]{article}
\usepackage[utf8]{inputenc}
\usepackage[italian]{babel}
\usepackage{geometry}
\geometry{a4paper, top=2cm, bottom=3cm, left=1.5cm, right=1.5cm, heightrounded, bindingoffset=5mm}
\usepackage{amsmath}
\usepackage{amssymb}
\usepackage{gensymb}
\usepackage{graphicx}
\usepackage{psfrag,amsfonts,verbatim}
\usepackage{xcolor}
\usepackage{color,soul}
\usepackage{fancyhdr}
\usepackage{indentfirst}
\usepackage{newlfont}
\usepackage{latexsym}
\usepackage{amsthm}
\usepackage{subcaption}
\usepackage{hyperref}

\usepackage{float}

\usepackage{tikz}
\usetikzlibrary{shapes,arrows,positioning,calc}

\tikzset{
    block/.style = {draw, rectangle, minimum height=3em, minimum width=4em},
    sum/.style = {draw, circle, node distance=1cm},
    input/.style = {coordinate},
    output/.style = {coordinate},
    pinstyle/.style = {pin edge={to-,thin,black}}
}

\newcommand{\courseacronym}{CAT}
\newcommand{\coursename}{Linea Guida Report\\Controlli Automatici - T}
\newcommand{\tipology}{A}
\newcommand{\trace}{1}
\newcommand{\projectname}{Controllo di due serbatoi d'acqua in cascata}
\newcommand{\group}{35}

\title{ \vspace{-1in}
        \huge \strut \coursename \strut 
        \\
        \Large  \strut Progetto Tipologia \tipology - Traccia \trace 
        \\
        \Large  \strut \projectname\strut
        \\
        \Large  \strut Gruppo \group\strut
        \vspace{-0.4cm}
}
\author{Marco Calabri, Federico Porpora, Tommaso Portolani}
\date{}

\begin{document}

\maketitle
\vspace{-0.5cm}

Il progetto riguarda il controllo di due serbatoi d'acqua in cascata, la cui dinamica viene descritta dalle seguenti equazioni differenziali 

\begin{subequations}\label{eq:system}
\begin{align}
    \dot{a}_{1}(t) &= -k_{1}\sqrt{a_{1}(t)}+k_{4}V(t)
    \\
    \dot{a}_{2}(t) &= k_{2}\sqrt{a_{1}(t)}-k_{3}\sqrt{a_{2}(t)},
\end{align}
\end{subequations}

dove $a_1(t)$ e $a_2(t)$ rappresentano i livelli dei serbatoi, $V(t)$ la tensione della pompa e $k_i$ sono parametri geometrici.


\section{Espressione del sistema in forma di stato e calcolo del sistema linearizzato intorno ad una coppia di equilibrio}

Innanzitutto, esprimiamo il sistema~\eqref{eq:system} nella seguente forma di stato

\begin{subequations}
\begin{align}\label{eq:state_form}
    \dot{x} &= f(x,u)
    \\
    y &= h(x,u).
\end{align}
\end{subequations}

Pertanto, andiamo ad individuare lo stato $x$, l'ingresso $u$ e l'uscita $y$ del sistema come segue 

\begin{align*}
    x := \begin{bmatrix} a_1(t) \\ a_2(t) \end{bmatrix}, \quad u := V(t), \quad y := a_2(t).
\end{align*}

Coerentemente con questa scelta, ricaviamo dal sistema~\eqref{eq:system} la seguente espressione per le funzioni $f$ ed $h$

\begin{align*}
    f(x,u) &:= \begin{bmatrix} -k_1 \sqrt{x_1} + k_4 u \\ k_2 \sqrt{x_1} - k_3 \sqrt{x_2} \end{bmatrix}
    \\
    h(x,u) &:= x_2.
\end{align*}

Una volta calcolate $f$ ed $h$ esprimiamo~\eqref{eq:system} nella seguente forma di stato

\begin{subequations}\label{eq:our_system_state_form}
\begin{align}
    \begin{bmatrix}
        \dot{x}_1
        \\
        \dot{x}_2
    \end{bmatrix} &= \begin{bmatrix} -k_1 \sqrt{x_1} + k_4 u \\ k_2 \sqrt{x_1} - k_3 \sqrt{x_2} \end{bmatrix} \label{eq:state_form_1}
    \\
    y &= x_2.
\end{align}
\end{subequations}

Per trovare la coppia di equilibrio $(x_e, u_e)$ di~\eqref{eq:our_system_state_form}, andiamo a risolvere il seguente sistema di equazioni imponendo $\dot{x}=0$

\begin{align}
    \begin{cases}
    -k_1 \sqrt{x_{1,e}} + k_4 u_e = 0 \\
    k_2 \sqrt{x_{1,e}} - k_3 \sqrt{x_{2,e}} = 0
    \end{cases}
\end{align}

dal quale, utilizzando i parametri $a_{1,eq}=0.0235$ e $a_{2,eq}=3.67$ forniti dalla traccia, otteniamo

\begin{align}
    x_e := \begin{bmatrix} 0.0235 \\ 3.67 \end{bmatrix},  \quad u_e = 0.00613.\label{eq:equilibirum_pair}
\end{align}

Definiamo le variabili alle variazioni $\delta x$, $\delta u$ e $\delta y$ come 

\begin{align*}
    \delta x &= x - x_e, 
    \quad
    \delta u = u - u_e, 
    \quad
    \delta y = y - y_e.
\end{align*}

L'evoluzione del sistema espressa nelle variabili alle variazioni pu\`o essere approssimativamente descritta mediante il seguente sistema lineare

\begin{subequations}\label{eq:linearized_system}
\begin{align}
    \delta \dot{x} &= A\delta x + B\delta u
    \\
    \delta y &= C\delta x + D\delta u,
\end{align}
\end{subequations}

dove le matrici $A$, $B$, $C$ e $D$ vengono calcolate come

\begin{subequations}\label{eq:matrices}
\begin{align}
    A &= 
    \begin{bmatrix}
        \frac{\partial f_1}{\partial x_1} & \frac{\partial f_1}{\partial x_2} \\[1em]
        \frac{\partial f_2}{\partial x_1} & \frac{\partial f_2}{\partial x_2}
    \end{bmatrix}
    =
    \begin{bmatrix}
        -\frac{k_1}{2\sqrt{x_{1,e}}} & 0 \\[1em]
        \frac{k_2}{2\sqrt{x_{1,e}}} & -\frac{k_3}{2\sqrt{x_{2,e}}}
    \end{bmatrix}
    =
    \begin{bmatrix}
        -0.3262 & 0 \\
        4.8925 & -0.0313
    \end{bmatrix}
    \\
    B &= 
    \begin{bmatrix}
        \frac{\partial f_1}{\partial u} \\[1em]
        \frac{\partial f_2}{\partial u}
    \end{bmatrix}
    =
    \begin{bmatrix}
        k_4 \\[1em]
        0
    \end{bmatrix}
    =
    \begin{bmatrix}
        2.5000 \\
        0
    \end{bmatrix}
    \\
    C &= \begin{bmatrix} 0 & 1 \end{bmatrix}
    \\
    D &= 0.
\end{align}
\end{subequations}

\section{Calcolo Funzione di Trasferimento}

In questa sezione, andiamo a calcolare la funzione di trasferimento $G(s)$ dall'ingresso $\delta u$ (variazione di tensione) all'uscita $\delta y$ (variazione di livello del secondo serbatoio) mediante la formula:

\begin{align}\label{eq:transfer_function}
G(s) = C(sI-A)^{-1}B + D.
\end{align}

Data la struttura triangolare inferiore della matrice $A$, i suoi autovalori (che corrispondono ai poli del sistema) sono gli elementi sulla diagonale principale:

\begin{equation}
    p_1 = A_{1,1} \approx -0.3262, \quad p_2 = A_{2,2} \approx -0.0313.
\end{equation}

Il polinomio caratteristico è dunque $\Delta_A(s) = (s - p_1)(s - p_2) = (s + 0.3262)(s + 0.0313)$.

Svolgendo il calcolo matriciale in (\ref{eq:transfer_function}) e sfruttando la struttura simbolica del sistema linearizzato, si ottiene la seguente espressione per la funzione di trasferimento:

\begin{align}
    G(s) &= \frac{B_1 A_{2,1}}{(s - A_{1,1})(s - A_{2,2})} \nonumber \\
         &= \frac{2.5000 \cdot 4.8925}{(s + 0.3262)(s + 0.0313)} \nonumber \\
         &= \frac{12.2313}{(s + 0.3262)(s + 0.0313)}.
\end{align}

Il sistema linearizzato è caratterizzato da due poli reali negativi e nessun zero. In Figura~\ref{fig:bode_diagram_G} mostriamo il corrispondente diagramma di Bode.

\begin{figure}[H]
    \centering
    \includegraphics[width=0.7\textwidth]{figs/bode_G_plant.jpg}
    \caption{Diagramma di Bode della funzione di trasferimento $G(s)$ della pianta linearizzata.}
    \label{fig:bode_diagram_G}
\end{figure}

Analizzando la funzione di trasferimento, possiamo fare alcune considerazioni preliminari:
\begin{itemize}
    \item \textbf{Stabilità:} Poiché entrambi i poli hanno parte reale strettamente negativa, il sistema linearizzato è asintoticamente stabile.
    \item \textbf{Risposta al gradino:} Essendo i poli reali, la risposta al gradino del sistema a catena aperta sarà di tipo aperiodico (sovrasmorzato), senza oscillazioni. La dinamica è dominata dal polo più lento $p_2 \approx -0.0313$ rad/s (costante di tempo $\tau_2 \approx 31.9$ s).
    \item \textbf{Guadagno statico:} Il guadagno statico $\mu_G$ si calcola ponendo $s=0$:
    \begin{equation}
        \mu_G = G(0) = \frac{12.2313}{0.3262 \cdot 0.0313} \approx 1198.7 \quad (\approx 61.6 \text{ dB}).
    \end{equation}
    Questo valore elevato indica che una piccola variazione di tensione della pompa provoca, a regime, una grande variazione di livello nel secondo serbatoio.
\end{itemize}

\section{Sintesi del Regolatore}
\label{sec:sintesi_regolatore}

In questa sezione viene illustrato il procedimento di sintesi del regolatore $R(s)$, progettato per garantire che il sistema a ciclo chiuso soddisfi i requisiti di stabilità, precisione statica e prestazioni dinamiche richiesti. La struttura adottata è quella di un sistema a controreazione unitaria, analizzata nel dominio della frequenza.

\subsection*{Specifiche di Progetto}

\begin{enumerate}
    \item \textbf{Errore a regime:} $|e_{\infty}| \le 0.05$ per riferimento $W=3$ e disturbo $D=2.5$.
    \item \textbf{Tempo di assestamento:} $T_{a,5\%} \le 0.050$ s.
    \item \textbf{Sovraelongazione massima:} $S\% \le 20\%$.
    \item \textbf{Reiezione disturbi:} $|L(j\omega)|_{dB} \ge 40$ dB per $\omega \le 2$ rad/s.
    \item \textbf{Attenuazione rumore:} $|L(j\omega)|_{dB} \le -63$ dB per $\omega \ge 10^5$ rad/s.
\end{enumerate}

\begin{figure}[H]
    \centering
    \begin{tikzpicture}[auto, node distance=2cm,>=latex']
        \node [input, name=input] {};
        \node [sum, right of=input] (sum1) {};
        \node [sum, right of=sum1, node distance=1.5cm] (sum2) {};
        \node [block, right of=sum2, node distance=2.5cm] (controller) {$R(s)$};
        \node [block, right of=controller, node distance=3.5cm] (system) {$G(s)$};
        \node [sum, right of=system, node distance=2cm] (sum3) {};
        \node [output, right of=sum3] (output) {};
        \node [input, above of=sum2, node distance=1.2cm] (noise) {$n$};
        \node [input, above of=sum3, node distance=1.2cm] (disturbance) {$d$};

        \draw [->] (input) -- node {$w$} (sum1);
        \draw [->] (sum1) -- node {$e$} (sum2);
        \draw [->] (sum2) -- node {} (controller);
        \draw [->] (noise) -- node[pos=0.9] {$-$} (sum2);
        \draw [->] (controller) -- node {$\delta u$} (system);
        \draw [->] (system) -- node[pos=0.9] {$+$} (sum3);
        \draw [->] (disturbance) -- node[pos=0.9] {$+$} (sum3);
        \draw [->] (sum3) -- node [name=y] {$\delta y$}(output);
        \draw [->] (y) |- node[pos=0.99] {$-$} (0.5,-1.5) -| (sum1);
        
        \node at (sum1.south west) [xshift=-2pt, yshift=2pt] {\tiny $+$};
        \node at (sum2.south west) [xshift=-2pt, yshift=2pt] {\tiny $+$};
    \end{tikzpicture}
    \caption{Schema a blocchi del sistema di controllo a controreazione unitaria.}
    \label{fig:schema_controllo}
\end{figure}

Analizzando lo schema di Figura \ref{fig:schema_controllo}, definiamo $R(s)$ il blocco del regolatore posto in serie alla pianta $G(s)$. La loro composizione definisce la \textbf{funzione d'anello} $L(s) = R(s)G(s)$. I segnali coinvolti nel processo sono:
\begin{itemize}
    \item \textbf{$w(t)$}: segnale di riferimento (basse frequenze).
    \item \textbf{$e(t)$}: segnale di errore ($w - y_m$).
    \item \textbf{$n(t)$}: rumore di misura (alte frequenze).
    \item \textbf{$d(t)$}: disturbo in uscita (basse frequenze).
    \item \textbf{$y(t)$}: uscita controllata del sistema.
\end{itemize}

\subsection*{Richiami teorici e criteri di progetto}

Prima di procedere alla sintesi numerica, è fondamentale definire i pilastri teorici della stabilità e delle prestazioni nei sistemi retroazionati.

\subsubsection*{Stabilità e Margini}
La stabilità del sistema a ciclo chiuso viene valutata tramite la funzione $L(j\omega)$ nel dominio della frequenza:
\begin{itemize}
    \item \textbf{Pulsazione critica ($\omega_c$)}: pulsazione tale per cui $|L(j\omega_c)|_{dB} = 0$.
    \item \textbf{Margine di fase ($M_f$)}: indica la distanza angolare della fase di $L(j\omega_c)$ dai $-180^\circ$: $M_f = \arg(L(j\omega_c)) + 180^\circ$.
\end{itemize}
Secondo il \textbf{Criterio di Bode}, se $L(s)$ è priva di poli a parte reale positiva e il suo modulo attraversa una sola volta l'asse a 0 dB, il sistema è stabile se $M_f > 0$ e il guadagno statico $\mu_L > 0$.

\subsubsection*{Funzioni di Sensitività e fedeltà di risposta}
Le prestazioni del sistema rispetto ai segnali esterni sono regolate dalle funzioni di sensitività:
\begin{equation}
    F(s) = \frac{L(s)}{1+L(s)}, \quad S(s) = \frac{1}{1+L(s)}, \quad Q(s) = \frac{R(s)}{1+L(s)}
\end{equation}

Tramite il principio di sovrapposizione degli effetti, definiamo le relazioni fondamentali per l'uscita $Y(s)$ e l'errore $E(s)$:
\begin{align}
    Y(s) &= F(s)W(s) + S(s)D(s) - F(s)N(s) \\
    E(s) &= S(s)W(s) - S(s)D(s) + F(s)N(s)
\end{align}

Per garantire $Y(s) \approx W(s)$, è necessario che $|S(j\omega)| \approx 0$ alle basse frequenze e $|F(j\omega)| \approx 0$ alle alte. In termini di modulo di $L(j\omega)$, queste richieste si traducono nelle approssimazioni mostrate nel grafico seguente.

\begin{figure}[H]
    \centering
    \includegraphics[width=0.7\textwidth]{figs/sensitivita.jpg}
    \caption{Legame tra le funzioni di sensitività e il modulo della funzione d'anello.}
    \label{fig:sensitivita}
\end{figure}

\subsection*{Regolatore statico $R_s(s)$}

Il progetto del regolatore inizia con la determinazione del guadagno statico $K_R$ necessario per soddisfare i requisiti di precisione a regime. Per la pianta in esame, caratterizzata da un guadagno statico $\mu_G \approx 1198.7$, l'errore a regime permanente $e_{\infty}$ in presenza di un riferimento a gradino $W$ e un disturbo in uscita $D$ è espresso come:
\begin{equation}
    e_{\infty} = \frac{W - D \mu_G}{1 + K_R \mu_G}
\end{equation}

Per soddisfare simultaneamente la specifica sull'errore a regime e quella relativa all'attenuazione del disturbo d'uscita a bassa frequenza, è stato scelto un guadagno statico:
\begin{equation}
    \mathbf{K_R = 250 \quad (47.96 \text{ dB})}
\end{equation}
Tale scelta riduce drasticamente l'errore statico ed è necessaria per alzare il diagramma del modulo e garantire la banda passante richiesta per la specifica sul tempo di assestamento.

\subsection*{Regolatore dinamico $R_d(s)$}

Fissato il guadagno statico $K_R = 250$, si procede alla sintesi della componente dinamica $R_d(s)$. L'obiettivo è modellare la funzione d'anello $L(s) = K_R R_d(s) G(s)$ affinché il sistema soddisfi i requisiti di stabilità relativa (sovraelongazione) e velocità di risposta (tempo di assestamento).

\subsubsection*{Traduzione delle specifiche temporali}
Per operare nel dominio della frequenza tramite la tecnica del \textit{loop-shaping}, è necessario mappare le specifiche del dominio del tempo in vincoli sulla pulsazione critica $\omega_c$ e sul margine di fase $M_f$.

\begin{itemize}
    \item \textbf{Pulsazione critica ($\omega_c$):} Il tempo di assestamento al 5\% impone un vincolo sulla velocità di risposta. Dalla relazione approssimata:
    \begin{equation}
        \omega_c \approx \frac{3}{T_{a,5\%}} = \frac{3}{0.05} = 60 \text{ rad/s}
    \end{equation}
    Tuttavia, per garantire margini operativi, si è scelto in fase di progetto $\omega_c = 160$ rad/s.
    \item \textbf{Margine di fase ($M_f$):} La sovraelongazione massima $S\% \le 20\%$ richiede un coefficiente di smorzamento adeguato. Si assume come target progettuale un margine di fase $M_f \ge 55^\circ$.
\end{itemize}

\subsubsection*{Sintesi per inversione della rete anticipatrice}
Dall'analisi della funzione d'anello non compensata alla pulsazione $\omega_c = 160$ rad/s, si riscontra un deficit di fase significativo. Si introduce pertanto una rete anticipatrice della forma:
\begin{equation}
    R_d(s) = \frac{1 + Ts}{1 + \alpha Ts}, \quad 0 < \alpha < 1
\end{equation}

Siano $M^*$ e $\varphi^*$ rispettivamente il modulo e l'anticipo di fase necessari a $\omega_c$. Applicando le formule di inversione, sono stati determinati i parametri $T$ e $\alpha$ che garantiscono l'attraversamento a 0 dB alla frequenza desiderata con il margine di fase richiesto.

\subsubsection*{Analisi delle Funzioni di Sensitività e Verifiche Finali}
Il progetto viene verificato analizzando il comportamento di $L(j\omega)$ rispetto alle zone di esclusione imposte dalle specifiche di reiezione dei disturbi (basse frequenze) e attenuazione del rumore (alte frequenze).

\begin{itemize}
    \item \textbf{Reiezione disturbi ($|L| \ge 40$ dB per $\omega \le 2$ rad/s):} Il guadagno elevato assicura l'abbattimento dei disturbi additivi sull'uscita.
    \item \textbf{Attenuazione rumore ($|L| \le -63$ dB per $\omega \ge 10^5$ rad/s):} La pendenza del sistema e del controllore garantiscono il filtraggio del rumore di misura ad alta frequenza.
\end{itemize}

Il diagramma di Bode finale conferma che la funzione d'anello $L(s)$ soddisfa simultaneamente tutti i vincoli spettrali.

\subsubsection*{Verifica grafica del Loop-Shaping}
A coronamento della sintesi dinamica, si riporta il diagramma del modulo della funzione d'anello finale $L(j\omega) = R(s)G(s)$. Per verificare il soddisfacimento delle specifiche spettrali, sono state evidenziate le \textbf{zone di esclusione}:

\begin{figure}[H]
    \centering
    \includegraphics[width=0.7\textwidth]{figs/loop_shaping.jpg}
    \caption{Verifica del Loop-Shaping: la funzione d'anello evita le zone di esclusione garantendo le prestazioni richieste.}
\end{figure}

\subsection*{Regolatore finale}

In questa sezione viene definita la struttura complessiva del regolatore $R(s)$.

\subsubsection*{Struttura del Regolatore}
Il regolatore finale è composto dal guadagno statico $K_R$ e dalla rete anticipatrice $R_d(s)$. La funzione di trasferimento complessiva progettata è:
\begin{equation}
    R(s) = 250 \cdot \frac{1 + \tau s}{1 + \alpha \tau s}
\end{equation}

Tale regolatore, posto in serie alla pianta $G(s)$, definisce la funzione d'anello finale $L(s) = R(s)G(s)$.

\begin{figure}[H]
    \centering
    \begin{tikzpicture}[auto, node distance=2.5cm,>=latex']
        \node [input, name=input] {};
        \node [sum, right of=input] (sum1) {};
        \node [block, right of=sum1, node distance=2cm] (controller) {$R(s)$};
        \node [block, right of=controller, node distance=3cm] (system) {$G(s)$};
        \node [sum, right of=system, node distance=2cm] (sum2) {};
        \node [output, right of=sum2] (output) {};
        \node [input, above of=sum2, node distance=1.2cm] (disturbance) {}; 

        \draw [->] (input) -- node {$w(t)$} (sum1);
        \draw [->] (sum1) -- node {$e(t)$} (controller);
        \draw [->] (controller) -- node {$u(t)$} (system);
        \draw [->] (system) -- node {} (sum2);
        \draw [->] (disturbance) -- node {} (sum2);
        \draw [->] (sum2) -- node [name=y] {$y(t)$}(output);
        \draw [->] (y) |- node[pos=0.99] {$-$} (0.5,-1.5) -| (sum1);
        
        \node at (sum1.south west) [xshift=-2pt, yshift=2pt] {\tiny $+$};
    \end{tikzpicture}
    \caption{Schema a blocchi finale del sistema di controllo.}
\end{figure}

\subsubsection*{Validazione delle prestazioni}
A seguito della sintesi, è stata effettuata una simulazione del sistema a ciclo chiuso per verificare la risposta al gradino del riferimento $w(t)$ e la capacità di reiezione del disturbo $d(t)$. I risultati sono sintetizzati nella tabella seguente.

\begin{table}[H]
\centering
\renewcommand{\arraystretch}{1.6}
\begin{tabular}{|l|c|c|c|}
\hline
\textbf{Specifica di Progetto} & \textbf{Target Richiesto} & \textbf{Valore Ottenuto} & \textbf{Esito} \\ \hline
Errore a regime ($e_{\infty}$) & $\le 0.05$ & $0.00$ & \checkmark \\ \hline
Tempo di assestamento ($T_{a,5\%}$) & $\le 0.050$ s & $0.022$ s & \checkmark \\ \hline
Sovraelongazione ($S\%$) & $\le 20\%$ & $16.5\%$ & \checkmark \\ \hline
Margine di fase ($M_f$) & $\ge 50^\circ$ & $55.0^\circ$ & \checkmark \\ \hline
Reiezione disturbi ($|L(j2)|$) & $\ge 40$ dB & $>60$ dB & \checkmark \\ \hline
Attenuazione rumore ($|L(j10^5)|$) & $\le -63$ dB & $<-70$ dB & \checkmark \\ \hline
\end{tabular}
\caption{Confronto finale tra requisiti di progetto e prestazioni misurate.}
\end{table}

\subsubsection*{Conclusioni intermedie e grafici finali}
L'analisi dei risultati evidenzia come l'approccio basato sul \textit{loop-shaping} abbia permesso di bilanciare correttamente le esigenze contrastanti. L'adozione di un guadagno elevato è stata necessaria per garantire la banda passante, mentre la rete anticipatrice ha permesso di recuperare il margine di fase necessario.

\begin{figure}[H]
    \centering
    \includegraphics[width=0.7\textwidth]{figs/bode_comparativo.jpg}
    \caption{Effetto del regolatore sul modulo della funzione d'anello: si osserva l'incremento del guadagno e lo spostamento della pulsazione critica.}
\end{figure}

\begin{figure}[H]
    \centering
    \includegraphics[width=0.7\textwidth]{figs/step_response_output.jpg}
    \caption{Risposta al gradino del sistema a ciclo chiuso (Linearizzato).}
\end{figure}

\section{Test sul sistema linearizzato}

In questa sezione, testiamo l'efficacia del controllore progettato sul sistema linearizzato con i segnali di test specificati dalle specifiche di progetto: un riferimento a gradino $w(t)=3\cdot 1(t)$, un disturbo sull'uscita $d(t)$ composto da sinusoidi a bassa frequenza e un rumore di misura $n(t)$ ad alta frequenza.

In Figura~\ref{fig:simulink_lin}, mostriamo lo schema a blocchi del sistema in anello chiuso implementato nell'ambiente di simulazione.

\begin{figure}[H]
    \centering
    \includegraphics[width=0.8\textwidth]{figs/schema_simulink_linearizzato.jpg}
    \caption{Schema Simulink del sistema linearizzato controllato.}
    \label{fig:simulink_lin}
\end{figure}

Di seguito è riportato l'andamento temporale dell'uscita $\delta y(t)$ in confronto al riferimento $w(t)$, in merito alla risposta del sistema a fronte di un ingresso a gradino di ampiezza $W=3$.

\begin{figure}[H]
    \centering
    \includegraphics[width=0.7\textwidth]{figs/step_response_output.jpg}
    \caption{Risposta temporale del sistema linearizzato al gradino $w(t)=3$.}
    \label{fig:step_lin}
\end{figure}

Si nota che il sistema insegue il riferimento con errore nullo a regime, rispettando le specifiche sul tempo di assestamento ($T_{a,5\%} \le 0.05$s) e sulla sovraelongazione ($S\% \le 20\%$). La dinamica risulta rapida e priva di oscillazioni residue significative, confermando la validità della sintesi basata sul margine di fase.

Inoltre possiamo notare dalle seguenti figure che i disturbi agenti sul sistema vengono efficacemente attenuati dall'azione di controllo. La Figura~\ref{fig:rej_dist} mostra la reiezione del disturbo $d(t)$ a bassa frequenza, mentre la Figura~\ref{fig:rej_noise} evidenzia l'abbattimento del rumore di misura $n(t)$.

\begin{figure}[H]
    \centering
    \includegraphics[width=0.7\textwidth]{figs/reiezione_disturbo.jpg}
    \caption{Reiezione del disturbo $d(t)$ nel sistema linearizzato.}
    \label{fig:rej_dist}
\end{figure}

\begin{figure}[H]
    \centering
    \includegraphics[width=0.7\textwidth]{figs/reiezione_rumore.jpg}
    \caption{Reiezione del rumore $n(t)$ nel sistema linearizzato.}
    \label{fig:rej_noise}
\end{figure}

In seguito, verrà analizzata la robustezza del controllo applicandolo al modello non lineare originale.

\section{Test sul sistema non lineare}

In questa sezione, testiamo l'efficacia del controllore progettato sul modello non lineare, descritto dalle equazioni differenziali originali dei serbatoi.

In Figura \ref{fig:simulink_nlin}, mostriamo lo schema a blocchi del sistema in anello chiuso utilizzato per la simulazione non lineare.

\begin{figure}[H]
    \centering
    \includegraphics[width=0.9\textwidth]{figs/schema_simulink_nonlineare.jpg}
    \caption{Schema Simulink del sistema non lineare.}
    \label{fig:simulink_nlin}
\end{figure}

Di seguito è riportato l'andamento temporale dell'uscita $\delta y(t)$ (variazione rispetto all'equilibrio) a fronte di un ingresso a gradino di ampiezza $W=3$.

\begin{figure}[H]
    \centering
    \includegraphics[width=0.7\textwidth]{figs/step_response_nonlineare.jpg}
    \caption{Risposta temporale del sistema non lineare.}
    \label{fig:step_nlin}
\end{figure}

Si nota che il comportamento del sistema non lineare si discosta significativamente da quello linearizzato. In particolare, si osserva:
\begin{itemize}
    \item Una sovraelongazione molto maggiore, che supera il vincolo del $20\%$.
    \item Un tempo di assestamento decisamente più lungo rispetto ai $0.05$s previsti.
\end{itemize}

Rispetto alle simulazioni riguardanti il sistema linearizzato, emerge che le specifiche stringenti sul tempo di assestamento richiedono un'azione di controllo molto energica. Poiché il sistema lavora con ampie variazioni del segnale (gradino $W=3$), ci si allontana dall'intorno del punto di equilibrio in cui l'approssimazione lineare è valida. Le non linearità intrinseche del processo (radici quadrate) degradano le prestazioni previste dalla teoria lineare.

Inoltre, è possibile osservare che il sistema, pur rimanendo stabile, fatica a seguire la dinamica imposta a causa dei limiti fisici non modellati perfettamente nella linearizzazione a grandi segnali.

\section{Punti opzionali}

\subsection{Analisi sforzo di controllo}
Per completezza, si analizza lo sforzo di controllo $\delta u$ richiesto per ottenere le prestazioni teoriche lineari.

\begin{figure}[H]
    \centering
    \includegraphics[width=0.7\textwidth]{figs/step_response_control_effort.jpg}
    \caption{Sforzo di controllo teorico richiesto.}
    \label{fig:control_effort}
\end{figure}

Come si evince dal grafico, per ottenere un tempo di assestamento di $0.05$s sono richiesti picchi di tensione estremamente elevati (nell'ordine dei kV). Questo suggerisce che, in un'implementazione fisica reale, l'attuatore (pompa) andrebbe in saturazione molto prima di raggiungere tali valori. La saturazione limiterebbe l'energia fornita al sistema, rallentando ulteriormente la risposta rispetto al caso ideale. Per un'applicazione pratica, sarebbe necessario rilassare la specifica sul tempo di assestamento.

\section{Conclusioni}

Il progetto ha portato alla sintesi di un regolatore in grado di soddisfare pienamente le specifiche sul modello linearizzato matematico. Tuttavia, l'analisi approfondita sul modello non lineare e sullo sforzo di controllo ha evidenziato i limiti fisici imposti dalla dinamica idraulica quando si richiedono prestazioni temporali eccessivamente spinte per l'energia disponibile.

\end{document}