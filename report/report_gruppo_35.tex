\documentclass[a4paper, 11pt]{article}
\usepackage[margin=3cm]{geometry}
\usepackage[]{fontenc}
\usepackage[utf8]{inputenc}
\usepackage[italian]{babel}
\usepackage{geometry}
\geometry{a4paper, top=2cm, bottom=3cm, left=1.5cm, right=1.5cm, heightrounded, bindingoffset=5mm}
\usepackage{amsmath}
\usepackage{amssymb}
\usepackage{gensymb}
\usepackage{graphicx}
\usepackage{psfrag,amsmath,amsfonts,verbatim}
\usepackage{xcolor}
\usepackage{color,soul}
\usepackage{fancyhdr}
\usepackage{indentfirst}
\usepackage{graphicx}
\usepackage{newlfont}
\usepackage{amssymb}
\usepackage{amsmath}
\usepackage{latexsym}
\usepackage{amsthm}
%\usepackage{subfigure}
\usepackage{subcaption}
\usepackage{psfrag}
\usepackage{footnote}
\usepackage{graphics}
\usepackage{color}
\usepackage{hyperref}
\usepackage{tikz}
\usepackage{float} % Aggiunto per posizionare le figure con [H]

\usetikzlibrary{snakes}
\usetikzlibrary{positioning}
\usetikzlibrary{shapes,arrows}

	
	\tikzstyle{block} = [draw, fill=white, rectangle, 
	minimum height=3em, minimum width=6em]
	\tikzstyle{sum} = [draw, fill=white, circle, node distance=1cm]
	\tikzstyle{input} = [coordinate]
	\tikzstyle{output} = [coordinate]
	\tikzstyle{pinstyle} = [pin edge={to-,thin,black}]

\newcommand{\courseacronym}{CAT}
\newcommand{\coursename}{Relazione Progetto\\Controlli Automatici - T}
\newcommand{\tipology}{A} % Inserisci tipologia corretta se diversa
\newcommand{\trace}{1}    % Inserisci traccia corretta se diversa
\newcommand{\projectname}{Controllo di due serbatoi d'acqua in cascata}
\newcommand{\group}{35}

%opening
\title{ \vspace{-1in}
		\huge \strut \coursename \strut 
		\\
		\Large  \strut Progetto Tipologia \tipology - Traccia \trace 
		\\
		\Large  \strut \projectname\strut
		\\
		\Large  \strut Gruppo \group\strut
		\vspace{-0.4cm}
}
\author{Marco Calabri, Federico Porpora, Tommaso Portolani}
\date{}

\begin{document}

\maketitle
\vspace{-0.5cm}

Il progetto riguarda il controllo di due serbatoi d'acqua in cascata, la cui dinamica viene descritta dalle seguenti equazioni differenziali 
%
\begin{subequations}\label{eq:system}
\begin{align}
	\dot{a}_{1}(t) &= -k_{1}\sqrt{a_{1}(t)}+k_{4}V(t) \\
    \dot{a}_{2}(t) &= k_{2}\sqrt{a_{1}(t)}-k_{3}\sqrt{a_{2}(t)},
\end{align}
\end{subequations}
%
dove $a_1(t)$ e $a_2(t)$ rappresentano i livelli dei serbatoi, $V(t)$ la tensione applicata alla pompa e $k_i$ sono parametri fisici del sistema.

\section{Espressione del sistema in forma di stato e calcolo del sistema linearizzato intorno ad una coppia di equilibrio}

Innanzitutto, esprimiamo il sistema~\eqref{eq:system} nella seguente forma di stato
%
\begin{subequations}
\begin{align}\label{eq:state_form}
	\dot{x} &= f(x,u)
	\\
	y &= h(x,u).
\end{align}
\end{subequations}
%
Pertanto, andiamo individuare lo stato $x$, l'ingresso $u$ e l'uscita $y$ del sistema come segue 
%
\begin{align*}
	x := \begin{bmatrix} a_1(t) \\ a_2(t) \end{bmatrix}, \quad u := V(t), \quad y := a_2(t).
\end{align*}
%
Coerentemente con questa scelta, ricaviamo dal sistema~\eqref{eq:system} la seguente espressione per le funzioni $f$ ed $h$
%
\begin{align*}
	f(x,u) &:= \begin{bmatrix} -k_1 \sqrt{x_1} + k_4 u \\ k_2 \sqrt{x_1} - k_3 \sqrt{x_2} \end{bmatrix}
	\\
	h(x,u) &:= x_2.
\end{align*}
%
Una volta calcolate $f$ ed $h$ esprimiamo~\eqref{eq:system} nella seguente forma di stato
%
\begin{subequations}\label{eq:our_system_state_form}
\begin{align}
	\begin{bmatrix}
		\dot{x}_1
		\\
		\dot{x}_2
	\end{bmatrix} &= \begin{bmatrix} -k_1 \sqrt{x_1} + k_4 u \\ k_2 \sqrt{x_1} - k_3 \sqrt{x_2} \end{bmatrix} \label{eq:state_form_1}
	\\
	y &= x_2.
\end{align}
\end{subequations}
%
Per trovare la coppia di equilibrio $(x_e, u_e)$ di~\eqref{eq:our_system_state_form}, andiamo a risolvere il seguente sistema di equazioni imponendo $\dot{x}=0$
%
\begin{align}
	\begin{cases}
    -k_1 \sqrt{x_{1,e}} + k_4 u_e = 0 \\
    k_2 \sqrt{x_{1,e}} - k_3 \sqrt{x_{2,e}} = 0,
    \end{cases}
\end{align}
%
dal quale otteniamo, utilizzando i dati della traccia ($a_{1,eq}=0.0235$ e $a_{2,eq}=3.67$),
%
\begin{align}
	x_e := \begin{bmatrix} 0.0235 \\ 3.67 \end{bmatrix},  \quad u_e = 0.00613.\label{eq:equilibirum_pair}
\end{align}
%
Definiamo le variabili alle variazioni $\delta x$, $\delta u$ e $\delta y$ come 
%
\begin{align*}
	\delta x &= x - x_e, 
	\quad
	\delta u = u - u_e, 
	\quad
	\delta y = y - y_e.
\end{align*}
%
L'evoluzione del sistema espressa nelle variabili alle variazioni pu\`o essere approssimativamente descritta mediante il seguente sistema lineare
%
\begin{subequations}\label{eq:linearized_system}
\begin{align}
	\delta \dot{x} &= A\delta x + B\delta u
	\\
	\delta y &= C\delta x + D\delta u,
\end{align}
\end{subequations}
%
dove le matrici $A$, $B$, $C$ e $D$ vengono calcolate come
%
\begin{subequations}\label{eq:matrices}
\begin{align}
	A &= \begin{bmatrix} -0.3262 & 0 \\ 4.8925 & -0.0313 \end{bmatrix}
	\\
	B &= \begin{bmatrix} 2.5000 \\ 0 \end{bmatrix}
	\\
	C &= \begin{bmatrix} 0 & 1 \end{bmatrix}
	\\
	D &= 0.
\end{align}
\end{subequations}
%
\section{Calcolo Funzione di Trasferimento}

In questa sezione, andiamo a calcolare la funzione di trasferimento $G(s)$ dall'ingresso $\delta u$ all'uscita $\delta y$ mediante la seguente formula 
%
%
\begin{align}\label{eq:transfer_function}
G(s) = C(sI-A)^{-1}B + D = \frac{12.23}{(s+0.3262)(s+0.0313)}.
\end{align}
%
Dunque il sistema linearizzato~\eqref{eq:linearized_system} è caratterizzato dalla funzione di trasferimento~\eqref{eq:transfer_function} con 2 poli $p_1 = -0.3262, p_2 = -0.0313$ e 0 zeri.
In Figura \ref{fig:bode_G} mostriamo il corrispondente diagramma di Bode.

\begin{figure}[H]
    \centering
    \includegraphics[width=0.7\textwidth]{figs/bode_G_plant.jpg}
    \caption{Diagramma di Bode della funzione di trasferimento $G(s)$.}
    \label{fig:bode_G}
\end{figure}

Inoltre, notiamo che il sistema è asintoticamente stabile (poli a parte reale negativa) e presenta un guadagno statico elevato ($\mu_G \approx 1198$).


\section{Mappatura specifiche del regolatore}
\label{sec:specifications}

Le specifiche da soddisfare sono
\begin{itemize}
	\item[1)] Errore a regime $|e_{\infty}| \le 0.05$ per riferimento $W=3$ e disturbo $D=2.5$.\\
	\item[2)] Tempo di assestamento al 5\%: $T_{a,5\%} \le 0.050$ s.\\
	\item[3)] Sovraelongazione massima: $S\% \le 20\%$.\\
	\item[4)] Margine di fase $M_f \ge 30^{\circ}$ (Target di progetto $55^{\circ}$).\\
	\item[5)] Reiezione disturbi sull'uscita: attenuazione di almeno 40 dB in banda [0, 2.0] rad/s.\\
	\item[6)] Reiezione rumore di misura: attenuazione di almeno 63 dB in banda $[10^5, 5\cdot10^6]$ rad/s.
\end{itemize}
%
Andiamo ad effettuare la mappatura punto per punto le specifiche richieste.
La specifica sul tempo di assestamento impone una pulsazione di attraversamento minima $\omega_c \approx 3/T_{a,5\%} = 60$ rad/s (sceglieremo $\omega_c = 160$ rad/s per margine di sicurezza). La specifica sulla sovraelongazione richiede un margine di fase adeguato. Le specifiche sui disturbi definiscono delle zone proibite nel diagramma del modulo di $L(s)$.

Pertanto, in Figura \ref{fig:loop_shaping}, mostriamo il diagramma di Bode della funzione di trasferimento $G(s)$ con le zone proibite emerse dalla mappatura delle specifiche (qui visualizziamo direttamente il risultato finale con i vincoli).

\begin{figure}[H]
    \centering
    \includegraphics[width=0.7\textwidth]{figs/loop_shaping.jpg}
    \caption{Loop Shaping con evidenziate le zone proibite (Disturbi e Rumore) e i target di fase.}
    \label{fig:loop_shaping}
\end{figure}

Si può notare che la pianta $G(s)$ da sola non soddisfa i requisiti di banda passante e non evita le zone proibite a bassa frequenza, necessitando quindi di un'azione di controllo.

\section{Sintesi del regolatore statico}
\label{sec:static_regulator}

In questa sezione progettiamo il regolatore statico $R_s(s)$ partendo dalle analisi fatte in sezione~\ref{sec:specifications}.
Per soddisfare la specifica sull'errore a regime e garantire sufficiente guadagno per la reiezione dei disturbi a bassa frequenza, calcoliamo il guadagno $K_R$.
Dai calcoli effettuati nel codice Matlab, si è scelto un guadagno statico $K_R = 250$.

Dunque, definiamo la funzione estesa $G_e(s) = R_s(s)G(s)$ e, in Figura \ref{fig:bode_comparativo}, mostriamo il suo diagramma di Bode per verificare se e quali zone proibite vengono attraversate.

\begin{figure}[H]
    \centering
    \includegraphics[width=0.7\textwidth]{figs/bode_comparativo.jpg}
    \caption{Confronto tra $G(s)$ e la funzione d'anello finale $L(s)$ (che include il guadagno statico).}
    \label{fig:bode_comparativo}
\end{figure}
 
Da Figura \ref{fig:bode_comparativo}, emerge che l'aumento di guadagno solleva la curva del modulo soddisfacendo le specifiche a bassa frequenza, ma non è sufficiente a garantire il margine di fase alla nuova pulsazione di attraversamento.

Inoltre, possiamo notare che il margine di fase del sistema col solo guadagno statico sarebbe insufficiente per le specifiche di sovraelongazione.


\section{Sintesi del regolatore dinamico}

In questa sezione, progettiamo il regolatore dinamico $R_d(s)$.
%
Dalle analisi fatte in Sezione~\ref{sec:static_regulator}, notiamo di essere nello Scenario di tipo B (necessità di anticipo di fase).
Dunque, progettiamo $R_d(s)$ riccorrendo a una rete anticipatrice della forma $R_d(s) = \frac{1+\tau s}{1+\alpha \tau s}$.
Utilizzando le formule di inversione per imporre un attraversamento a $\omega_c = 160$ rad/s con il margine di fase desiderato, abbiamo ottenuto i parametri $\tau$ e $\alpha$ riportati nello script Matlab.


In Figura \ref{fig:loop_shaping_final}, mostriamo il diagramma di Bode della funzione d'anello $L(s) = R_d(s) G_e(s)$

\begin{figure}[H]
    \centering
    \includegraphics[width=0.7\textwidth]{figs/loop_shaping.jpg}
    \caption{Diagramma di Bode della funzione d'anello $L(s)$ finale.}
    \label{fig:loop_shaping_final}
\end{figure}

Possiamo notare che la funzione d'anello passa attraverso la "cruna dell'ago" definita dalle specifiche, garantendo stabilità, banda passante e reiezione dei disturbi.

\section{Test sul sistema linearizzato}

In questa sezione, testiamo l'efficacia del controllore progettato sul sistema linearizzato con simulazioni temporali.

In Figura \ref{fig:simulink_lin}, mostriamo lo schema a blocchi del sistema in anello chiuso utilizzato per la simulazione lineare (comprensivo di disturbi $d(t)$ e $n(t)$).
% Assicurati di avere l'immagine dello schema linearizzato, altrimenti usa un placeholder o genera lo screenshot
\begin{figure}[H]
    \centering
    \includegraphics[width=0.9\textwidth]{figs/schema_simulink_linearizzato.jpg} % Sostituisci con il nome file corretto se diverso
    \caption{Schema Simulink per la simulazione del sistema linearizzato.}
    \label{fig:simulink_lin}
\end{figure}

Di seguito è riportato l'andamento dell'uscita $\delta y$ in confronto al riferimento, in merito alla risposta del sistema a fronte di un ingresso a gradino di ampiezza $W=3$.

\begin{figure}[H]
    \centering
    \includegraphics[width=0.7\textwidth]{figs/step_response_output.jpg}
    \caption{Risposta al gradino del sistema linearizzato.}
    \label{fig:step_lin}
\end{figure}

Si nota che il sistema rispetta perfettamente le specifiche: tempo di assestamento inferiore a $0.05$s, sovraelongazione contenuta entro il $20\%$ ed errore a regime nullo.

Inoltre possiamo notare dalle seguenti figure che i disturbi $d(t)$ e il rumore $n(t)$ vengono efficacemente abbattuti.

\begin{figure}[H]
    \centering
    \includegraphics[width=0.7\textwidth]{figs/reiezione_disturbo.jpg}
    \caption{Reiezione del disturbo $d(t)$ sull'uscita.}
    \label{fig:rej_dist}
\end{figure}

\begin{figure}[H]
    \centering
    \includegraphics[width=0.7\textwidth]{figs/reiezione_rumore.jpg}
    \caption{Attenuazione del rumore di misura $n(t)$.}
    \label{fig:rej_noise}
\end{figure}

\section{Test sul sistema non lineare}

In questa sezione, testiamo l'efficacia del controllore progettato sul modello non lineare con le equazioni costitutive originali dei serbatoi.

In Figura \ref{fig:simulink_nlin}, mostriamo lo schema a blocchi del sistema in anello chiuso.

\begin{figure}[H]
    \centering
    \includegraphics[width=0.9\textwidth]{figs/schema_simulink_nonlineare.jpg} % Sostituisci con nome file schema non lineare
    \caption{Schema Simulink del sistema non lineare.}
    \label{fig:simulink_nlin}
\end{figure}

Di seguito è riportato l'andamento dell'uscita $\delta y(t)$ in confronto al riferimento, in merito alla risposta del sistema a fronte di un ingresso a gradino $W=3$.

\begin{figure}[H]
    \centering
    \includegraphics[width=0.7\textwidth]{figs/step_response_nonlineare.jpg} % Usa l'immagine dello scope non lineare
    \caption{Risposta al gradino del sistema non lineare.}
    \label{fig:step_nlin}
\end{figure}

Si nota che il comportamento del sistema non lineare si discosta da quello linearizzato. In particolare, si osserva una sovraelongazione maggiore e un tempo di assestamento più lungo.

Rispetto alle simulazioni riguardanti il sistema linearizzato emerge che le specifiche stringenti sul tempo di assestamento ($0.05$s) richiedono un'azione di controllo molto energica.

Inoltre, è possibile osservare che l'elevata tensione richiesta dal controllore porta il sistema a lavorare in una regione fortemente non lineare, dove l'approssimazione lineare perde validità per ampie variazioni del segnale, causando il degrado delle prestazioni osservato.


\section{Punti opzionali}

\subsection{Primo punto}
Per completezza, si segnala che lo sforzo di controllo $\delta u$ richiesto per ottenere le prestazioni teoriche è estremamente elevato (picchi nell'ordine dei kV), il che suggerisce che in un'implementazione fisica reale sarebbe necessario saturare l'attuatore o rilassare la specifica sul tempo di assestamento.

\subsection{Secondo punto}
...

\section{Conclusioni}

Il progetto ha portato alla sintesi di un regolatore in grado di soddisfare pienamente le specifiche sul modello linearizzato. Tuttavia, l'analisi sul modello non lineare ha evidenziato i limiti fisici imposti dalla dinamica del sistema idraulico quando si richiedono prestazioni temporali molto spinte.

\end{document}