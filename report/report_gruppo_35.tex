\documentclass[a4paper, 11pt]{article}
\usepackage[margin=3cm]{geometry}
\usepackage[]{fontenc}
\usepackage[utf8]{inputenc}
\usepackage[italian]{babel}
\usepackage{geometry}
\geometry{a4paper, top=2cm, bottom=3cm, left=1.5cm, right=1.5cm, heightrounded, bindingoffset=5mm}
\usepackage{amsmath}
\usepackage{amssymb}
\usepackage{gensymb}
\usepackage{graphicx}
\usepackage{psfrag,amsmath,amsfonts,verbatim}
\usepackage{xcolor}
\usepackage{color,soul}
\usepackage{fancyhdr}
\usepackage{indentfirst}
\usepackage{graphicx}
\usepackage{newlfont}
\usepackage{amssymb}
\usepackage{amsmath}
\usepackage{latexsym}
\usepackage{amsthm}
%\usepackage{subfigure}
\usepackage{subcaption}
\usepackage{psfrag}
\usepackage{footnote}
\usepackage{graphics}
\usepackage{color}
\usepackage{hyperref}
\usepackage{tikz}


\usetikzlibrary{snakes}
\usetikzlibrary{positioning}
\usetikzlibrary{shapes,arrows}

	
	\tikzstyle{block} = [draw, fill=white, rectangle, 
	minimum height=3em, minimum width=6em]
	\tikzstyle{sum} = [draw, fill=white, circle, node distance=1cm]
	\tikzstyle{input} = [coordinate]
	\tikzstyle{output} = [coordinate]
	\tikzstyle{pinstyle} = [pin edge={to-,thin,black}]

\newcommand{\courseacronym}{CAT}
\newcommand{\coursename}{Linea Guida Report\\Controlli Automatici - T}
\newcommand{\tipology}{A} % Inserito A (o lascia vuoto se diverso)
\newcommand{\trace}{1}    % Inserito Traccia 1
\newcommand{\projectname}{Controllo di due serbatoi d'acqua in cascata}
\newcommand{\group}{X} % Inserisci il numero del tuo gruppo

%opening
\title{ \vspace{-1in}
		\huge \strut \coursename \strut 
		\\
		\Large  \strut Progetto Tipologia \tipology - Traccia \trace 
		\\
		\Large  \strut \projectname\strut
		\\
		\Large  \strut Gruppo \group\strut
		\vspace{-0.4cm}
}
\author{Autore 1, Autore 2, Autore 3} % Inserisci i vostri nomi
\date{}

\begin{document}

\maketitle
\vspace{-0.5cm}

Il progetto riguarda il controllo di due serbatoi d'acqua in cascata, la cui dinamica viene descritta dalle seguenti equazioni differenziali 
%
\begin{subequations}\label{eq:system}
\begin{align}
	\dot{a}_{1}(t) &= -k_{1}\sqrt{a_{1}(t)}+k_{4}V(t)
	\\
	\dot{a}_{2}(t) &= k_{2}\sqrt{a_{1}(t)}-k_{3}\sqrt{a_{2}(t)},
\end{align}
\end{subequations}
%
dove $a_1(t)$ e $a_2(t)$ rappresentano i livelli dei serbatoi, $V(t)$ la tensione della pompa e $k_i$ sono parametri geometrici.


\section{Espressione del sistema in forma di stato e calcolo del sistema linearizzato intorno ad una coppia di equilibrio}

Innanzitutto, esprimiamo il sistema~\eqref{eq:system} nella seguente forma di stato
%
\begin{subequations}
\begin{align}\label{eq:state_form}
	\dot{x} &= f(x,u)
	\\
	y &= h(x,u).
\end{align}
\end{subequations}
%
Pertanto, andiamo individuare lo stato $x$, l'ingresso $u$ e l'uscita $y$ del sistema come segue 
%
\begin{align*}
	x := \begin{bmatrix} a_1(t) \\ a_2(t) \end{bmatrix}, \quad u := V(t), \quad y := a_2(t).
\end{align*}
%
Coerentemente con questa scelta, ricaviamo dal sistema~\eqref{eq:system} la seguente espressione per le funzioni $f$ ed $h$
%
\begin{align*}
	f(x,u) &:= \begin{bmatrix} -k_1 \sqrt{x_1} + k_4 u \\ k_2 \sqrt{x_1} - k_3 \sqrt{x_2} \end{bmatrix}
	\\
	h(x,u) &:= x_2.
\end{align*}
%
Una volta calcolate $f$ ed $h$ esprimiamo~\eqref{eq:system} nella seguente forma di stato
%
\begin{subequations}\label{eq:our_system_state_form}
\begin{align}
	\begin{bmatrix}
		\dot{x}_1
		\\
		\dot{x